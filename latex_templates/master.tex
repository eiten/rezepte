\documentclass[a4paper,12pt]{scrartcl}

% Basic encoding and languages
\usepackage{fontspec}
\usepackage[ngerman]{babel}
\usepackage[style=swiss]{csquotes} % Swiss quotes

% Layout
\usepackage[left=2.5cm, right=2.5cm, top=3cm, bottom=3cm]{geometry}
\usepackage{tabularx}
\usepackage{booktabs}
\usepackage{parskip} % No indentation for paragraphs

% Science & Units
\usepackage[version=4]{mhchem}
\usepackage{siunitx}
\sisetup{
  locale = DE,
  detect-all,
  range-phrase = --,
  range-units = single
}

% Define custom units if needed (Example mappings from DB)
\DeclareSIUnit\EL{EL}
\DeclareSIUnit\TL{TL}
\DeclareSIUnit\Prise{Prise}
\DeclareSIUnit\Stk{Stk.}

% Metadata
\title{ << recipe.name >> }
\author{ << recipe.author >> }
\date{}

\begin{document}
\noindent
{\huge \textbf{<< recipe.name >>}} \\[0.3cm]
\noindent
{\large Von << recipe.author >>}

\vspace{1cm}

\noindent
\begin{tabularx}{\textwidth}{@{} p{5cm} X @{}}
    \toprule
    \textbf{Zutaten} & \textbf{Zubereitung} \\
    \midrule

    <% for step in steps %>
        % --- Ingredients Column ---
        \vspace{0pt} % Hack to align top
        <% if step.ingredients %>
            <% for ing in step.ingredients %>
                % Logic: If it is a real unit, use \qty, else simple text formatting
                <% if ing.unit_type == 'si' %>
                    <% if ing.amount_max %>
                        \qtyrange{<< ing.amount_min >>}{<< ing.amount_max >>}{<< ing.unit_latex >>}
                    <% else %>
                        \qty{<< ing.amount_min >>}{<< ing.unit_latex >>}
                    <% endif %>
                <% else %>
                    % Text units (EL, TL, Stk)
                    \num{<< ing.amount_min >>}
                    <% if ing.amount_max %>--\num{<< ing.amount_max >>}<% endif %>
                    << ing.unit_symbol >>
                <% endif %>
                
                << ing.item >>
                <% if ing.note %> \textit{(<< ing.note >>)} <% endif %>
                \newline
            <% endfor %>
        <% endif %>
        &
        % --- Instructions Column ---
        \vspace{0pt} % Hack to align top
        << step.latex_text >> \\
        \addlinespace[1em]
    <% endfor %>

    \bottomrule
\end{tabularx}

\vfill
\footnotesize{Generiert am << current_date >> mit Rezept Manager}

\end{document}
