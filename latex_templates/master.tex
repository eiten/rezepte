% FIX: 'footsepline' enforces the footer line globally and reliably
\documentclass[a4paper,12pt]{scrartcl}

% -- LuaLaTeX Setup --
\usepackage{fontspec}
\usepackage[ngerman]{babel}
\usepackage[style=swiss]{csquotes} 

% -- Font Setup (Source Sans 3) --
% Using \setmainfont changes the body text.
% Using \setsansfont ensures headers (like "Zubereitung") match.
\setmainfont{source-sans-3-latin-300-normal.ttf}[
    Path = << font_path >>,
    BoldFont = source-sans-3-latin-400-normal.ttf,
    ItalicFont = source-sans-3-latin-300-italic.ttf,
    BoldItalicFont = source-sans-3-latin-400-italic.ttf
]

% Add this to fix \textbf looking like Computer Modern
\setsansfont{source-sans-3-latin-300-normal.ttf}[
    Path = << font_path >>,
    BoldFont = source-sans-3-latin-400-normal.ttf,
    ItalicFont = source-sans-3-latin-300-italic.ttf,
    BoldItalicFont = source-sans-3-latin-400-italic.ttf
]

% Set the entire document to sans-serif to achieve the Frutiger aesthetic
\renewcommand{\familydefault}{\sfdefault}
% Ensure KOMA-Script headers also use the new font family
\addtokomafont{disposition}{\rmfamily}

% -- PHOSPHOR ICONS SETUP (The elegant way) --
% Load the font directly from the folder.
% "Ligatures=TeX" ensures that words like "Warning" become icons.
\newfontfamily\iconFont[
    Path = << font_path >>,
    Extension = .ttf,
    Renderer = HarfBuzz,
    UprightFont = Phosphor-Light, % Default is Light
    BoldFont = Phosphor           % Bold switch uses Regular
]{Phosphor}

\newcommand{\picon}[1]{\raisebox{-1.1pt}{\iconFont\symbol{"#1}}}

% -- Layout & Tables --
\usepackage[
    left=2cm,
    right=2cm,
    top=1.5cm,
    bottom=1.8cm,
    footskip=0.5cm,
    includehead,
]{geometry}
\usepackage{xltabular} % Replacement for tabularx to allow page breaks 
\usepackage{booktabs}
\usepackage{parskip} 
\usepackage{array} 
\usepackage{xcolor}

% -- Header/Footer Setup --
\usepackage{scrlayer-scrpage}
\pagestyle{scrheadings}
\clearpairofpagestyles 

% Content for page 2 and onwards 
\ihead[]{\headmark} % Recipe name on the left
\ohead[]{Seite \thepage} % Page number on the right
\automark{section}% -- Layout & Tables --

\usepackage{xltabular} % Replacement for tabularx to allow page breaks 
\usepackage{booktabs}
\usepackage{parskip} 
\usepackage{array} 
\usepackage{xcolor}

% -- Header/Footer Setup --
\usepackage{scrlayer-scrpage}
\usepackage{ccicons}

\pagestyle{scrheadings}
\clearpairofpagestyles 

% Content for page 2 and onwards 
\ihead[]{\footnotesize \headmark} % Recipe name on the left
\ohead[]{\footnotesize Seite \thepage} % Page number on the right
\automark{section}

\atbegindocument{\thispagestyle{plain}}

% Footer content for all pages
\ifoot[\footnotesize << base_url >>]{\footnotesize << base_url >>}
\cfoot[\ccbyncsa]{\ccbyncsa}
\ofoot[\footnotesize Stand: << date_version >>]{\footnotesize Letzte Änderung: << date_version >>}

% -- Science & Units --
\usepackage{siunitx}
\sisetup{
  locale = DE, 
  detect-all,
  range-phrase = --,
  range-units = single,
  per-mode = symbol
}

% -- Custom Units from Database --
<< unit_defs >>

% -- Metadata --
\title{ << recipe.name >> }
\author{ << recipe.author >> }
\date{}

\begin{document}

% -- Header --
\section*{<< recipe.name >>}
% This line ensures the recipe name appears in \headmark for page 2+
\markboth{<< recipe.name >>}{<< recipe.name >>}

\noindent
Von << recipe.author >>
<% if source -%>
    \hfill {\small \textit{Quelle: << source >>}}
<% endif -%>

% -- Preamble --
<% if preamble -%>
    \vspace{0.5em}

    \noindent
    << preamble >>
<% endif -%>

% -- Content Table (with page breaks) --
\noindent
\begin{xltabular}{\textwidth}{@{} >{\raggedright\arraybackslash}p{3.5cm} X @{}}
    % Header for the very first page
    \textbf{Zutaten} & \textbf{Zubereitung} \\
    \midrule
    \endfirsthead
    
    % Header for all following pages 
    \textbf{Zutaten} & \textbf{Zubereitung} \\
    \midrule
    \endhead
    <% for step in steps -%>
        <% if step.is_ingredients == 1 -%>
            % -- CASE A: Normal ingredient list --
            <% if step.ingredients -%>
                <% for ing in step.ingredients -%>
                    <% if ing.amount_min -%>
                        % -- Case 1: Quantity exists --
                        <% if ing.amount_max -%>
                            \qtyrange{<< ing.amount_min >>}{<< ing.amount_max >>}{<< ing.unit_cmd >>}
                        <% else -%>
                            \qty{<< ing.amount_min >>}{<< ing.unit_cmd >>}
                        <% endif -%>
                        ~\textbf{<< ing.item >>}
                    <% else -%>
                        % -- Case 2: No quantity (e.g. Bittermandelaroma) --
                        \textbf{<< ing.item >>}
                    <% endif -%>
                    <% if ing.note -%> \textit{\small{(<< ing.note >>)}}<% endif -%>
                    <% if not loop.last -%>\newline<% endif -%> 
                <% endfor -%>
            <% endif -%>
        
        <% else -%>
            \hfill \raisebox{-4pt}{\iconFont\large{\symbol{<< step.latex_icon >>}}}
        <% endif -%>
        <% if not loop.last -%>\rule[-1em]{0pt}{0pt}<% endif -%>
        &
        << step.latex_text >>
        <% if not loop.last -%>\rule[-1em]{0pt}{0pt}\\<% endif -%>
    <% endfor -%>
\end{xltabular}

\end{document}